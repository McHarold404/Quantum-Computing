\chapter{Getting Started}
\section{Introduction}
Quantum mechanics is the branch of physics that governs the world of elementary particles such as protons and electrons, and it is paradoxical, unintuitive, and radically strange.It is impossible to know the whole state of a quantum system because the very act of measuring the state(by which we get to know the state) disturbs the system. Instead of using the wave-particle duality to define the state of elementary particles, it is better to say that they behave in a “quantum mechanical” way.Particles do not have trajectories, but rather take all paths simultaneously (in superposition). And on top of that Richard Feynman infamously said "I think I can safely say that nobody understands quantum mechanics!. Some of the points worth noting about quantum mechanics are:
\begin{itemize}
  \item The superposition principle explains how a particle can be superimposed between two states at the same time.
  \item The measurement principle tells us how measuring a particle changes its state, and how much information we can access from a particle.
  \item   The unitary evolution axiom governs how the state of the quantum system evolves in time.
\end{itemize}

\subsection{ Birth of Quantum Mechanics}
Until quite recently, the evidence strongly favored wave-like propagation. Diffraction of light, a wave interference phenomenon, was observed in 1655 by Grimaldi. A successful theory of wave-like light propagation, due to Huygens, was developed in 1678. Then, a major breakthrough came from Young’s double-slit experiment. Then experiments like the Photoelectric effect and Black body radiations came long which were found inconsistent with the wave’s nature of light. These phenomena were then explained by considering light as particles containing discrete packets of energy, called photons.

\subsection{Young's Double Slit Experiment}
The interference pattern observed on the screen from the two slits was explained appropriately by the wave’s nature of light.
Now consider placing a photo detector at the viewing screen, an bring down the intensity to the level that it only records the arrival of a photon occasionally. Initially, we would observe  that as we turn down the intensity of the source, the magnitude of each click remains constant, but the time between successive clicks increases. From here, we can infer  that light is emitted from the source as discrete particles (photons) — the intensity of light is proportional to the rate at which photons are emitted by the source. And since you turned the intensity of the light source down sufficiently, it only emits a photon once every few seconds. Now when a photon is emitted from a source, a question worth asking is where will the particle be detected on the screen. In other words, we can look at it from a probabilistic sense that what is the probability that the photon is detected on the screen as a function of x which is the distance from the mean position. Now, when only a single slit is opened, we observe that the probabilities are directly in sync with the interference pattern observed from a single slit. Intuitively, one would guess that if both the slits are opened, the probabilities of a photon being detected will directly be the sum of the probabilities observed when one of the two slits are opened separately. But, what we observe is the probabilities observed are again in sync with the interference pattern(intensities) observed when both the slits are open. This is where the particle nature, fails to explain the behaviour of particles . The nature of the contradiction can be seen even more clearly at “dark” points x, where the probability of detection is 0 when both slits are open, even though it is non-zero if either slit is open. This truly defies reason! Particles do not have trajectories, but rather take all paths simultaneously (in superposition).
Explanation of this phenomenon: We posit that instead of taking a single path from the source to the screen, it has a probabilistic amplitude A1(y) with which, it goes through slit 1 and A2(y) with which, it travels through the second slit. There is no specific path decided for the photon, but the photons follow a “superposition of both the paths”. In essence, the exact path of the photon is unknown.

\subsection {EPR Paradox and  Bell's inequality}
In 1935, Albert Einstein, Boris Podolsky, and Nathan Rosen developed a thought experiment to demonstrate what they felt was a lack of completeness in quantum mechanics. A principal feature of quantum mechanics is that not all the classical physical quantities can be defined with unlimited precision. They believed that there must exist a different set of observables that give qualitatively different but complete and accurate descriptions of a quantum mechanical system. A perfect analogy for this can be given by defining it by an experiment.
Consider the examples of tossing a coin. For all common purposes, we believe that the outcome of a coin toss is completely random-heads and tails with equal probability. Now, one could argue that if one knew all the parameters like position, momentum, then we could use Newton’s law to calculate which side will face up. One more way to say this is that this coin flip amplifies our lack of knowledge about the system which makes the output completely random. Similarly, Einstein believed that the randomness of quantum measurements reflected our lack of knowledge about additional degrees of freedom, or what Einstein called “hidden variables”, of the quantum system. He gave this famous statement in regard to this experiment, “God does not play dice” arguing that there was an objective truth that was undiscovered in the field of quantum mechanics that didn’t rely on probabilistic measurements.

Bell came up with the following experiment to test the EPR paradox. Let us assume that two particles are produced in the Bell state $|\psi+\rangle$  in a laboratory, and then fly in opposite directions to two distant laboratories. Upon arrival, each of the two qubits is subject to one of two measurements. The decision about which of the two experiments is to be performed at each lab is made randomly at the last moment, so that speed of light considerations rule out information about the choice at one lab being transmitted to the other. The measurements are cleverly chosen to distinguish between the predictions of quantum mechanics and any local hidden variable theory. Concretely, the experiment measures the correlation between the outcomes of the two experiments. The choice of measurements is such that any classical hidden variable theory predicts that the correlation between the two outcomes can be at most 0.75, whereas quantum mechanics predicts that the correlation will be at most $cos^2 \pi/8≈0.85$. Thus the experiment allows us to distinguish between the predictions of quantum mechanics and any local hidden variable theory! Therefore this ruled out any theories about a hidden variable and verified the fact that the qubits/electrons are in-fact, in a superposition of states complete information about their orientation can not be known.